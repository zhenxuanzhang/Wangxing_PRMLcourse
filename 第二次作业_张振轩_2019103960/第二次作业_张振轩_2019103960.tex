\documentclass[11.5pt]{article}
%[11.5pt]为设置字体大小为11.5磅

\usepackage{ctex} 
%显示中文
\usepackage{amsmath} 
%目的是为改善信息结构,并且改善打印输出包含数学公式的文档
\usepackage{mathtools}
\usepackage{commath}
\usepackage{extarrows}

\usepackage{amsthm}
%这个package是定理环境专用的
\usepackage{natbib}
%自带参考文献宏包
\usepackage{mathrsfs}
%用于产生一种数学用的花体字
\usepackage{amssymb}
%宏包套件AMSFonts中的一个宏包,调用该宏包时,AMSFonts也被调用,定义了msam和mabm中的全部数学符号命令
\usepackage{graphicx}
\usepackage{color}

\addtolength{\textwidth}{1in}
%改变长度的命令是 \addtolength 和 \setlength。latex可认的是 cm, mm, in 和 pt. 变量可设为负数。
%\textwidth: 文本的宽度
%in 英寸(1in=2.54cm)
%pt 点(1in=72.27pt)
%bp 大点(1in= 72bp)
%\columnsep: 列间距
%\topmargin: 页眉到页边的距离
%\topskip: 页眉与正文的距离
%\textheight: 正文的高度
%\textwidth: 文本的宽度
%\oddsidemargin: 奇数页的左面页边距
%\evensidemargin : 偶数页的左面页边距
\addtolength{\oddsidemargin}{-0.5in}
\addtolength{\textheight}{1.4in}
\addtolength{\topmargin}{-0.6in}


\renewcommand{\baselinestretch}{1.1}
%\newcommand defines a new command, and makes an error if it is already defined.
%\renewcommand redefines a predefined command, and makes an error if it is not yet defined.
%\providecommand defines a new command if it isn't already defined,  or does nothing if it exists.

%\newcommand{\qed}{{\unskip\nobreak\hfil\penalty50\hskip2em\vadjust{}
%            \nobreak\hfil$\Box$\parfillskip=0pt\finalhyphendemerits=0\par}}
\renewcommand{\theequation}{\thesection.\arabic{equation}}

\newtheorem{thm}{Theorem}[section] %(If you want theorem numbered使用下面语句来定于定理名字和输出显示。
%\newtheorem{name名字}{Printed output输出显示}
\newtheorem{lemma}{Lemma}[section] %%    with section number.
\newtheorem{cor}{Corollary}[section]
\newtheorem{prop}{Proposition}[section]
\newtheorem{definition}{Definition}[section]
\newtheorem*{remark}{Remark}%带序号的定理




\usepackage{hyperref}
%在LaTeX(及一系列)中添加带超链接的交叉引用(超链接)
\hypersetup{
    colorlinks=true,%彩色链接
    linkcolor=blue,%内部链接颜色
    filecolor=magenta,%文件链接颜色
    urlcolor=cyan,%网页与电子邮件链接颜色
}



\title{Assignment \uppercase\expandafter{\romannumeral2}}
\author{Zhenxuan Zhang}
\CTEXoptions[today=old]

\begin{document}

\maketitle
\section{Problem 1.3}
	设取到一个苹果的概率为P(A),则有:
\begin{align*}
P(A) &= P(r)\cdot P(A|r) + P(b)\cdot P(A|b) +P(g)\cdot P(A|g)  \\
&= \begin{multlined}[t]
0.2\times \frac{3}{10} + 0.2\times \frac12 +0.6\times \frac{3}{10}
\end{multlined}\\
&= \begin{multlined}[t]
0.06+0.1+0.18
\end{multlined}\\
&= 0.34
\end{align*}

	设取到一个橘子的概率为P(B),则有:
\begin{align*}
P(B) &= P(r)\cdot P(B|r) + P(b)\cdot P(B|b) +P(g)\cdot P(B|g)  \\
&= \begin{multlined}[t]
0.2\times \frac{4}{10} + 0.2\times \frac12 +0.6\times \frac{3}{10}
\end{multlined}\\
&= 0.36
\end{align*}

由贝叶斯公式:
\begin{align*}
P(g|B) &= \frac{P(B|g)\cdot P(g)}{P(B)}  \\
&= \begin{multlined}[t]
\frac{0.6\times \frac{3}{10}}{0.36} 
\end{multlined}\\
&= 0.5
\end{align*}

\section{Problem 1.6}
 因为$x,y$相互独立,所以有:
 \[
 p(x,y) = p_x(x)\cdot p_y(y)
 \]

\begin{align*}
\implies
\iint xyp(x,y) \,\mathrm{d}x \mathrm{d}y &= 
\iint xyp_x(x)\cdot p_y(y) \,\mathrm{d}x \mathrm{d}y \\
&= \begin{multlined}[t]
\left( \int xp_x(x) \,\mathrm{d}x \right) \left(\int yp_y(y) \,\mathrm{d}y \right)
\end{multlined}\\
\implies
E_{x,y}(xy) &= E(x)\cdot E(y) \\
\implies
cov(x,y) &= E_{x,y}(xy) - E(x)\cdot E(y) = 0
\end{align*}

\section{Problem 1.8}
  
\begin{equation*}
\begin{split}
E(x) &= \int_{-\infty}^{+\infty} x\frac1{ \sqrt{2\pi\sigma^2}} 
exp \left\{ -\frac{1}{2\sigma^2} (x-\mu)^2 \right\} \,\mathrm{d}x \\
&\xlongequal{\text{令}\frac{x-\mu}{\sqrt{2\sigma^2}} = t }
\frac{\sqrt{2}\sigma}{\sqrt{2\pi\sigma^2}}\int_{-\infty}^{+\infty} (\sqrt{2}\sigma t + \mu) exp\{-t^2\} \,\mathrm{d}t \\
&= \frac1{\sqrt\pi}\sqrt{\pi}\mu \\
&=\mu
\end{split}
\end{equation*}


\begin{equation*}
\begin{split}
var(x) &= \int_{-\infty}^{+\infty} \left(x-E(x)\right)\frac1{ \sqrt{2\pi\sigma^2}} 
exp \left\{ -\frac{1}{2\sigma^2} (x-\mu)^2 \right\} \,\mathrm{d}x \\
&\xlongequal{\text{令}\frac{x-\mu}{\sqrt{2\sigma^2}} = t }
\frac{4\sigma^2}{\sqrt{\pi}} \int_{0}^{+\infty} 
t^2 \mathrm{e}^{-t^2} \,\mathrm{d}t \\
&\xlongequal{\text{令}t^2= u} \frac{2\sigma^2}{\sqrt{\pi}} \int_0^{+\infty} u^{\frac12} \mathrm{e}
^{-u} \,\mathrm{d}u \\
&=\frac{2\sigma^2}{\sqrt{\pi}} \Gamma(\frac32) \\
&= \sigma^2
\end{split}
\end{equation*}

\begin{equation*}
	\begin{split}
	\implies
	E(x^2) &= var(x) + [E(x)]^2\\
	&=\mu^2 + \sigma^2
	\end{split}
\end{equation*}






\end{document}









