\documentclass[12pt]{article}
\usepackage{ctex}
\usepackage{amsmath}
\usepackage{mathtools}
\usepackage{commath}
\usepackage{extarrows}

\title{数学模式}
\author{张振轩}
\date{\today} 

\begin{document}
	
\maketitle
\section{概说}
\subsection{行内公式}
行内公式一般在前后单个美元符号\$...\$表示,例如:\\
交换律是$a+b=b+a$,如$1+2=2+1=3$。\\
交换律是
\[a+b=b+a,\]
如
\[
1+2=2+1=3
\]

\subsection{自动编号}
latex提供带自动编号的数学公式,可以用equation环境表示。
\begin{equation}
	a+b=b+a \label{eq:commutative}
\end{equation}

\subsection{数学公式中插入文字}
使用amsmath提供的\\text命令 \\
$\text{被减数} - \text{减数} = \text{差}$
	
\section{数学结构}
\subsection{上标与下标}
上标用特殊字符\^表示,下标用\_表示

$A_{ij} = 2^{i+j}$

$A_i^k = B^k_i$ \qquad
$K_{n_i}  = K_{2^i} = 2^{2^i}$ \qquad
$3^{3^{3^{\cdot^{\cdot^{\cdot^3}}}}}$


数学公式中的撇号'就是一种特殊下标,用符号prime作上标

$a = a'$,$b_0' = b_0''$

用符号circ的上标表示角度

$A = 90^\circ$

行间公式多数数学算子的上下标,位置是在正上或正下方,

\[
	\max_n f(n) = \sum_{i=0}^n A_i
\]

但对积分号等个别算子,显示公式中的上下标也在右上右下角

%\[
%\int_0^1 f(t) \dif dt
%= \iint_D g(x,y) \dif dx \dif dy \]


在行内公式中,所有算子的上下标都在角标的位置

例如$\max_n f(n) = \sum_{i=0}^n A_i$

前面的上下标用mathtools宏包的prescript{上标}{下标}{元素}

$\prescript{n}{m}{H}_i^j$

\subsection{上下画线与花括号}

用overline和underline命令

$\overline{a+b} = 
\overline{a} + \overline{b}$ \\
$\underline{a} = (a_0,a_1,...,a_n)$

加箭头

$\overleftarrow{a+b}$\\
$\underleftrightarrow{a-b}$

$\vec{x} = \overrightarrow{AB}$

花括号

$\overbrace{a+b+c} = \underbrace{1+2+3}$

\[
(\overbrace{a_0,a_1,\dots,a_n}^{\text{共 $n+1$ 项}} )
 = ( \underbrace{0,0,\dots, 0}_{n}, 1)
 \]


\subsection{分式}
用frac\{分子\}\{分母\}

\[
\frac{1}{2} + \frac{1}{a} = \frac{a+2}{2a}
\]



二项式系数用binom

\[
(a+b)^2 = \binom{2}{0} a^2
+ \binom{2}{1} ab + \binom{2}{2} b^2
\]

\subsection{根式}

用sqrt

$\sqrt 4 = \sqrt[3]{8} = 2$

\subsection{矩阵}
矩阵环境matrix,pmatrix,bmatrix,vmatrix

不同列用符号\&分隔,行用换行符分隔

\[ A = \begin{pmatrix}
a_{11} & a_{12} & a_{13} \\
0 & a_{22}  & a_{23}  \\
0 & 0 & a_{a_{33}}
\end{pmatrix}
\]



\section{符号与类型}

\subsection{字母表与普通符号}
数学常数e使用罗马体的$\mathrm{e}$

虚数单位i也是$\mathrm{i}$

\subsection{数学算子}

\[
\mathcal{F}(x) = \sum_{k=0}^{\infty}
	\oint_o^1 f_k(x,t) \,\mathrm{d}t
	\]

注意积分式的写法,积分式中的微元dt里面,微分算子d应该使用直立罗马体,后面的变量则仍是默认的意大利体,并且用\,与签名的被积函数分开:

\[
 \int f(x) \,\mathrm{d}x
 \]
 
不带上下限的数学算子名

$\log,\lg,\ln,\sin,\cos,\tan,\exp$

带上下限的数学算子名

$\lim,\max,\min,\inf,\det,\varlimsup,\varliminf$

\begin{equation}
	\varlimsup_{k\to\infty} A_k = \lim_{J\to\infty}
	\lim_{K\to\infty}
	\bigcap_{j=1}^J \bigcup_{k=j}^K A_k
\end{equation}

\subsection{二元运算符与关系符}

$\times,\div,\cap,\cup,\bullet,\pm,\cdot,
\star,\ast,\setminus$

$\ne,\le,\ge,\in,\notin,\ll,\approx,\sim,\equiv,\subset\perp,\mid,\propto$


逻辑命令\\
$x=y \implies x+a = y+a$ \\
$x=y \impliedby x+a = y+a$ \\
$x=y \iff x\le y \And x\ge y$ \\	
	
$\forall x, \forall s$
	
	
\subsection{括号与定界符}

$\lvert, \rvert, \lVert, \rVert$
	
$\vert,\Vert$	

可变大小的定界符用left和right命令得到,他们分别把作为其参数的定界符转换为开符号和闭符号,同时按中间内容的高度自动调节大小。\\	
\[
\partial_x \partial_y \left[
\frac12 \left( x^2+y^2 \right)^2 +xy
\right]\]

left和right命令必须在同一行配对,但用来配对的定界符不需要与原来的是同一种括号,甚至可以使用一个句号.表示空的定界符。

\[
 \left.
 \int_0^x f(t, \lambda) \,\mathrm{d}t
 \right|_{x=1}, \qquad
 \lambda \in
 \left[\frac12,\infty\right).
 \]	

还有一个middle命令,在left和right中间再加一个定界符

\[
\Pr \left( X>\frac12
\middle\vert Y=0 \right)
= \left.
\int_0^1 p(t)\,\mathrm{d}t
\middle/ ( N^2+1) \right.
\]


\section{多行公式}

\subsection{罗列多个公式}
equation环境里面的换行命令是无效的,输入多行数学公式是使用gather环境

\begin{gather}
	a+b = b+a \\
	ab = ba
\end{gather}

\begin{gather*}
	2+5 = 4+3 \\
	3\times 6 = 2\times 9
\end{gather*}

在编号的多行公式中,可以在这一行的换行符之前使用notag命令阻止行编号


\begin{gather}
	1+2 = 3 \notag \\
	1+3 = 4 \notag \\
	2+3 = 5
\end{gather}

\subsection{拆分单个公式}
split环境并不开始一个数学公式,它用在equation,gather等数学环境中,可以把单个公式拆分成多行,其不产生编号,编号仍由外面的数学环境产生

\begin{equation}
	\begin{split}
	\cos 2x &= \cos^2 x - \sin^2 x \\
			&= 2\cos^2 x - 1
	\end{split}
\end{equation}


multline环境是equation环境的分行版本,可以使用换行符换行,
各行对齐方式是第一行左对齐最后一行右对齐中间部分居中
\begin{multline}
	1-2+3 \\
	2+4+5\\
	45+78+90
\end{multline}

\subsection{将公式组合成块}

\begin{align*}
	2^5 &= (1+1)^5 \\
	&= \begin{multlined}[t]
	\binom{5}{0}\cdot 1^5 +\binom{5}{1}\cdot 1^4 \cdot 1 + \binom{5}{2}\cdot 1^3 \cdot 1^2 
	+ \cdots
	\end{multlined}\\
	&= \binom{5}{0}+ \cdots
\end{align*}


\begin{align*}
P(A) &= P(r)\cdot P(A|r) + P(b)\cdot P(A|b) +P(g)\cdot P(A|g)  \\
&= \begin{multlined}[t]
0.2\times \frac{3}{10} + 0.2\times \frac12 +0.6\times \frac{3}{10}
\end{multlined}\\
&= \begin{multlined}[t]
0.06+0.1+0.18
\end{multlined}\\
&= 0.34
\end{align*}

\[
E(x) = \int_{-\infty}^{+\infty} x\frac1{ \sqrt{2\pi\sigma^2}} 
exp \left\{ -\frac{1}{2\sigma^2} (x-\mu)^2 \right\} \,\mathrm{d}x \\
\xlongequal{\frac{x-\mu}{\sqrt{2\sigma^2}} = t } \\
\frac{\sqrt{2}\sigma}{\sqrt{2\pi\sigma^2}}\int_{-\infty}^{+\infty} (\sqrt{2}\sigma t + \mu) exp\{-t^2\} \,\mathrm{d}t\\
= \frac1{\sqrt\pi}\sqrt{\pi}\mu
\]



	
	
	
\end{document}